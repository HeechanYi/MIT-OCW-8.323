\documentclass[11pt, a4paper]{article}
\usepackage[left=2.5cm,right=2.5cm,top=3.5cm,bottom=3.5cm,a4paper]{geometry}
\usepackage[utf8]{inputenc}
\usepackage{color}
\usepackage{kotex}
\usepackage{helvet}
\usepackage{amsmath, amssymb, amsthm, mathtools}
\setlength{\parindent}{0pt}
\usepackage{color}
\usepackage{array}
\newcolumntype{P}[1]{>{\centering\arraybackslash}p{#1}}
\renewcommand{\baselinestretch}{1.15}	                 % for title box horizontal space
\usepackage{tabularx}				  	 % for better rules in the table
\usepackage{enumitem}					 % for custom enumerate
\usepackage{minted}				         % for source-code with math formulas
\usepackage{romannum}

\newenvironment{tinyitem}
{\begin{itemize}[label={\tiny\(\bullet\)}]}
{\end{itemize}}

\newlist{numlist}{enumerate}{10}
\newcommand*\parenthesize[1]{(#1)}
\setlist[numlist]{label*=.\arabic*,format=\parenthesize,leftmargin=2em}
\setlist[numlist,1]{label=\arabic*}

\newlist{alphlist}{enumerate}{10}
\setlist[alphlist]{label*=.\arabic*,format=\parenthesize,leftmargin=2em}
\setlist[alphlist,1]{label=\alph*}

\newcommand{\newproblem}[2]
{\vspace{24pt} \Large{\textbf{#1.}} ({\it #2 points})}

\title{Assignment \#1}
\author{Heechan Yi}
\makeatletter 			% for author, title reference

\begin{document}
    \begin{tabularx}{\textwidth}{|X|}
	\hline
	\textsf{\bf MIT OCW 8.323: Quantum Field Theory \Romannum{1}} \hfill \\\
		
	\hfil {\LARGE \sf \@title \small{(\today)} } \hfil \\\

	{\hfill \@author} \\
        \hline
    \end{tabularx}
\\
\\
%%%%%%%%%%%%%%%%%%%%%%%%%%%%%%%%%%%%%%%%
%%%%%%%%%%%%%%%Problem 1%%%%%%%%%%%%%%%%
%%%%%%%%%%%%%%%%%%%%%%%%%%%%%%%%%%%%%%%%
\textbf{1. Review : Quantum Harmonic Oscillator in the Heisenberg Picture}\\
\textcolor{blue}{
Consider the Hamiltonian for a unit mass harmonic oscillator with frequency $\omega$
\begin{equation}\label{eqn:1.1}
    H = \frac{1}{2} \left( \hat{p}^2 + \omega^2\hat{x}^2 \right)
\end{equation}
In the Heisenberg picture $\hat{p}(t)$ and $\hat{x}(t)$ are dynamical variables which evolve with time. They obey the equal-time commutation relation
\begin{equation}\label{eqn:1.2}
    [\hat{x}(t), \hat{p}(t)] = i
\end{equation}
Here and below we set $\hbar = 1$.
}
%%%%%%%%%%%%%%%%%%%%%%%%%%%%%%%%%%%%%%%%
%%%%%%%%%%%%%Problem 1.(a)%%%%%%%%%%%%%%
%%%%%%%%%%%%%%%%%%%%%%%%%%%%%%%%%%%%%%%%
\begin{enumerate}
    \item [(a)] \textcolor{blue}{Obtain the Heisenberg evolution equation for $\hat{x}(t)$ and $\hat{p}(t)$.}
\end{enumerate}
%%%%%%%%%%%%%%%%%%%%%%%%%%%%%%%%%%%%%%%%%%%%%%%%%%%%%%%%%%%%%%%%%%%%%%%%%%%%%%%%
From Heisenberg equation of motion, we can know that an operator in Heisenberg picture follows below:
$$
\frac{d}{dt}\hat{A}(t) = i[H, \hat{A}(t)]
$$
So, $\hat{x}(t)$ is :
\begin{align}
    \frac{d}{dt}\hat{x}(t) = i[H, \hat{x}(t)] & = i \left( \frac{1}{2} [\hat{p}^2+\omega^2 x^2, \hat{x}] \right) \\
    & = \frac{i}{2} [\hat{p}^2, \hat{x} ] \\
    & = \frac{i}{2} \left(  \hat{p}[\hat{p}, \hat{x}] + [\hat{p}, \hat{x}]\hat{p} \right) \\
    & = \frac{i}{2} \times (-2i\hat{p}) \\
    & = \hat{p}(t)
\end{align}
And $\hat{p}(t)$ is:
\begin{align}
    \frac{d}{dt}\hat{p}(t) = i[H, \hat{p}(t)] & = i \left( \frac{1}{2} [\hat{p}^2+\omega^2 x^2, \hat{p}] \right) \\
    & = \frac{i}{2} [\omega^2\hat{x}^2, \hat{p} ] \\
    & = \frac{i}{2} \omega^2 \left( \hat{x}[\hat{x}, \hat{p}] + [\hat{x}, \hat{p}]\hat{x} \right) \\
    & = \frac{i}{2} \omega^2 \times (2i\hat{x}) \\
    & = - \omega^2 \hat{x}(t)
\end{align}
%%%%%%%%%%%%%%%%%%%%%%%%%%%%%%%%%%%%%%%%
%%%%%%%%%%%%%Problem 1.(b)%%%%%%%%%%%%%%
%%%%%%%%%%%%%%%%%%%%%%%%%%%%%%%%%%%%%%%%
\begin{enumerate}
    \item [(b)] \textcolor{blue}{Suppose the initial condition at $t=0$ are given by 
    \begin{equation}\label{eqn:1.3}
        \hat{x}(0) = \hat{x},\quad \hat{p}(0) = \hat{p}
    \end{equation} find $\hat{x}(t)$ and $\hat{p}(t)$.
    }
\end{enumerate}
%%%%%%%%%%%%%%%%%%%%%%%%%%%%%%%%%%%%%%%%%%%%%%%%%%%%%%%%%%%%%%%%%%%%%%%%%%%%%%%%
From Pb.1.(a), we solve that
\begin{align}
    \frac{d}{dt}\hat{x}(t) = \hat{p}(t) \\
    \frac{d}{dt}\hat{p}(t) = - \omega^2 \hat{x}(t)
\end{align}
Combining the two differential equation to solve $\hat{x}(t)$, we can summarize it:
\begin{equation}
    \frac{d^2}{dt^2} x(t) = - \omega^2x(t)
\end{equation}
Then we can guess a general solution of $\hat{x}(t)$:
\begin{equation}
    \hat{x}(t) = Ae^{i \omega t}+Be^{-i\omega t}
\end{equation}
such that
\begin{equation}
    \hat{p}(t) = i A \omega e^{i \omega t}-i B \omega e^{-i \omega t}
\end{equation}
Given initial condition (Eqn.(\ref{eqn:1.3}))
\begin{align}
    A+B = \hat{x} \\
    i\omega(A - B) = \hat{p}
\end{align}
Then,
\begin{align}
    A = \frac{1}{2}\left( \hat{x} + \frac{\hat{p}}{i\omega} \right)\\
    A = \frac{1}{2}\left( \hat{x} - \frac{\hat{p}}{i\omega} \right)
\end{align}
So, $\hat{x}(t)$ and $\hat{p}(t)$ are summarized
\begin{align}
    \hat{x}(t) & = \frac{1}{2}\left( \left( \hat{x} + \frac{\hat{p}}{i\omega} \right) e^{i \omega t} + \left( \hat{x} - \frac{\hat{p}}{i\omega} \right) e^{-i \omega t} \right) & = \hat{x}\cos \omega t + \frac{\hat{p}}{\omega}\sin{\omega t} \\
    \hat{p}(t) & = \frac{i \omega}{2}\left( \left(\hat{x} + \frac{\hat{p}}{i\omega} \right) e^{i \omega t} - \left(\hat{x} - \frac{\hat{p}}{i\omega} \right) e^{-i \omega t} \right) & = -\omega \hat{x}\sin \omega t + \hat{p}\cos{\omega t}
\end{align}

\newpage
%%%%%%%%%%%%%%%%%%%%%%%%%%%%%%%%%%%%%%%%
%%%%%%%%%%%%%Problem 1.(c)%%%%%%%%%%%%%%
%%%%%%%%%%%%%%%%%%%%%%%%%%%%%%%%%%%%%%%%
\begin{enumerate}
    \item [(c)] \textcolor{blue}{
    It is convenient to introduce operators $\hat{a}(t)$, $\hat{a}^\dagger(t)$ defined by
    \begin{equation}\label{eqn:1.4}
        \hat{x}(t) = \sqrt{\frac{1}{2\omega}}\left( \hat{a}(t) + \hat{a}^\dagger(t) \right), \quad \hat{p}(t) = -i\sqrt{\frac{\omega}{2}}\left( \hat{a}(t) - \hat{a}^\dagger(t) \right)
    \end{equation}
    Show that $\hat{a}(t)$ and $\hat{a}^\dagger(t)$ satisfying equal-time commutation relation
    \begin{equation}\label{eqn:1.5}
        [\hat{a}(t), \hat{a}^\dagger(t)] = 1
    \end{equation}
    }
\end{enumerate}
%%%%%%%%%%%%%%%%%%%%%%%%%%%%%%%%%%%%%%%%%%%%%%%%%%%%%%%%%%%%%%%%%%%%%%%%%%%%%%%%
From the definition(Eqn.(\ref{eqn:1.4})), we can rewrite that:
\begin{align}
    \hat{a}(t) + \hat{a}^\dagger(t) = \sqrt{2\omega} \hat{x}(t) \\
    \hat{a}(t) - \hat{a}^\dagger(t) = i\sqrt{\frac{2}{\omega}}\hat{p}(t)
\end{align}
then, we can write down $\hat{a}(t)$ and $\hat{a}^\dagger(t)$ in terms of $\hat{x}(t)$ and $\hat{p}(t)$:
\begin{align}
    \hat{a}(t) & = \frac{1}{\sqrt{2\omega}} \left( \omega \hat{x}(t) + i\hat{p}(t) \right) \\
    \hat{a}^\dagger(t) & = \frac{1}{\sqrt{2\omega}} \left( \omega \hat{x}(t) - i\hat{p}(t) \right)
\end{align}
Using the commutation relation in Eqn.(\ref{eqn:1.2}):
\begin{align}
    [\hat{a}(t), \hat{a}^\dagger(t)] & = \frac{1}{2\omega} [\omega \hat{x}(t) + i\hat{p}(t), \omega \hat{x}(t) - i\hat{p}(t)] \\
    & = \frac{1}{2\omega}\left( [\omega \hat{x}(t),  - i\hat{p}(t)] + [i\hat{p}(t), \omega \hat{x}(t)]\right) \\
    & = \frac{1}{2\omega} \times 2 \omega \\
    & = 1
\end{align}

\newpage
%%%%%%%%%%%%%%%%%%%%%%%%%%%%%%%%%%%%%%%%
%%%%%%%%%%%%%Problem 1.(d)%%%%%%%%%%%%%%
%%%%%%%%%%%%%%%%%%%%%%%%%%%%%%%%%%%%%%%%
\begin{enumerate}
    \item [(d)] \textcolor{blue}{
    Express the Hamiltonian in terms of $\hat{a}(t)$ and $\hat{a}^\dagger(t)$.
    }
\end{enumerate}
%%%%%%%%%%%%%%%%%%%%%%%%%%%%%%%%%%%%%%%%%%%%%%%%%%%%%%%%%%%%%%%%%%%%%%%%%%%%%%%%
Firstly, expand $\hat{x}^2(t)$ and $\hat{p}^2(t)$ in terms of $\hat{a}(t)$ and $\hat{a}^\dagger(t)$:
\begin{align}
    \hat{x}^2(t) & = \frac{1}{2\omega} \left( \hat{a}^2(t) + \hat{a}(t)\hat{a}^\dagger(t) + \hat{a}^\dagger(t)\hat{a}(t) + \hat{a}^{\dagger^2}(t)\right) \\
    \omega^2 \hat{x}^2(t) & = \frac{\omega}{2} \left( \hat{a}^2(t) + \hat{a}(t)\hat{a}^\dagger(t) + \hat{a}^\dagger(t)\hat{a}(t) + \hat{a}^{\dagger^2}(t)\right)
\end{align}
and
\begin{equation}
    \hat{p}^2(t) = -\frac{\omega}{2} \left( \hat{a}^2(t) - \hat{a}(t)\hat{a}^\dagger(t) - \hat{a}^\dagger(t)\hat{a}(t) + \hat{a}^{\dagger^2}(t)\right)
\end{equation}
Using the commutation relation from Pb. 1.(c) Eqn.(\ref{eqn:1.5}), we know that
\begin{equation}
    [\hat{a}(t), \hat{a}^\dagger(t)] = 1 \Rightarrow \hat{a}(t)\hat{a}^\dagger(t) -\hat{a}^\dagger(t)\hat{a}(t) = 1 \Rightarrow \hat{a}(t)\hat{a}^\dagger(t) = \hat{a}^\dagger(t)\hat{a}(t) + 1
\end{equation}
Therefore, the Hamiltonian(Eqn.(\ref{eqn:1.1})) can be expressed:
\begin{align}
    H = \frac{1}{2} (\hat{p}^2 + \omega^2 \hat{x}^2) & = \frac{\omega}{4} \left( 2(\hat{a}(t)\hat{a}^\dagger(t) +\hat{a}^\dagger(t)\hat{a}(t)) \right)\\
    & = \frac{\omega}{2}\left( 2\hat{a}^\dagger(t)\hat{a}(t) + 1 \right)\\
    & = \omega \left( \hat{a}^\dagger(t)\hat{a}(t) + \frac{1}{2} \right)
\end{align}

%%%%%%%%%%%%%%%%%%%%%%%%%%%%%%%%%%%%%%%%
%%%%%%%%%%%%%Problem 1.(e)%%%%%%%%%%%%%%
%%%%%%%%%%%%%%%%%%%%%%%%%%%%%%%%%%%%%%%%
\begin{enumerate}
    \item [(e)] \textcolor{blue}{
    Obtain the Heisenberg equations for $\hat{a}(t)$ and $\hat{a}^\dagger(t)$.
    }
\end{enumerate}
%%%%%%%%%%%%%%%%%%%%%%%%%%%%%%%%%%%%%%%%%%%%%%%%%%%%%%%%%%%%%%%%%%%%%%%%%%%%%%%%
Same process as solving Pb. 1(a),
\begin{align}
    \frac{d}{dt} \hat{a}(t) = i[H, \hat{a}(t)] \\
    \frac{d}{dt} \hat{a}^\dagger(t) = i[H, \hat{a}^\dagger(t)]
\end{align}
Then, using expasion of the Hamiltonian $H$ in terms of $\hat{a}(t)$ and $\hat{a}^\dagger(t)$ and the commutation relation from Eqn.(\ref{eqn:1.5})
\begin{align}
    \frac{d}{dt} \hat{a}(t) & = i\left[\omega \left( \hat{a}^\dagger(t)\hat{a}(t) + \frac{1}{2} \right), \hat{a}(t)\right] & = i\omega [\hat{a}^\dagger(t)\hat{a}(t), \hat{a}(t)] \\
    \frac{d}{dt} \hat{a}^\dagger(t) & = i\left[\omega \left( \hat{a}^\dagger(t)\hat{a}(t) + \frac{1}{2} \right), \hat{a}^\dagger(t)\right] & = i\omega [\hat{a}^\dagger(t)\hat{a}(t), \hat{a}^\dagger(t)]
\end{align}
rewriting the commutation,
\begin{align}
    \frac{d}{dt} \hat{a}(t) & = i\omega(\hat{a}^\dagger(t)\hat{a}(t)\hat{a}(t) - \hat{a}(t)\hat{a}^\dagger(t)\hat{a}(t)\hat{a}(t)) & = i\omega[\hat{a}^\dagger(t), \hat{a}]\hat{a} & = -i\omega\hat{a}(t)\\
    \frac{d}{dt} \hat{a}^\dagger(t) & = i\omega(\hat{a}^\dagger(t)\hat{a}(t)\hat{a}^\dagger(t) - \hat{a}^\dagger(t)\hat{a}^\dagger(t)\hat{a}(t)) & = i\omega \hat{a}^\dagger(t) [\hat{a}(t), \hat{a}^\dagger(t)] & = i\omega \hat{a}^\dagger(t)
\end{align}

\newpage

%%%%%%%%%%%%%%%%%%%%%%%%%%%%%%%%%%%%%%%%
%%%%%%%%%%%%%Problem 1.(f)%%%%%%%%%%%%%%
%%%%%%%%%%%%%%%%%%%%%%%%%%%%%%%%%%%%%%%%
\begin{enumerate}
    \item [(f)] \textcolor{blue}{
    Suppose the initial condition at $t = 0$ are given by
    \begin{equation}\label{eqn:1.6}
        \hat{a}(0) = \hat{a},\quad \hat{a}^{\dagger}(0) = \hat{a}^{\dagger}
    \end{equation}
    find $\hat{a}(t)$ and $\hat{a}^{\dagger}(t)$
    }
\end{enumerate}
%%%%%%%%%%%%%%%%%%%%%%%%%%%%%%%%%%%%%%%%%%%%%%%%%%%%%%%%%%%%%%%%%%%%%%%%%%%%%%%%
Same process as solving Pb. 1(b), using the result of Pb. 1(e)
\begin{align}
    \frac{d}{dt}\hat{a}(t) = -i\omega \hat{a}(t) \Rightarrow \hat{a}(t) = \hat{a}e^{-iwt} \\
    \frac{d}{dt}\hat{a}^{\dagger}(t) = -i\omega \hat{a}^{\dagger}(t) \Rightarrow \hat{a}^{\dagger}(t) = \hat{a}^{\dagger}e^{iwt}
\end{align}

%%%%%%%%%%%%%%%%%%%%%%%%%%%%%%%%%%%%%%%%
%%%%%%%%%%%%%Problem 1.(g)%%%%%%%%%%%%%%
%%%%%%%%%%%%%%%%%%%%%%%%%%%%%%%%%%%%%%%%
\begin{enumerate}
    \item [(g)] \textcolor{blue}{
    Express $\hat{x}(t), \hat{p}(t)$ and the Hamiltonian $H$ in terms of $\har{a}$ and $\hat{a}^\dagger$.
    }
\end{enumerate}
%%%%%%%%%%%%%%%%%%%%%%%%%%%%%%%%%%%%%%%%%%%%%%%%%%%%%%%%%%%%%%%%%%%%%%%%%%%%%%%%
We can use the definition of $\hat{x}(t), \hat{p}(t)$(Eqn. (\ref{eqn:1.5})) in Prob. 1.(c) and replace the result of $\hat{a}(t)$, $\hat{a}^{\dagger}(t)$ into terms of $\hat{a}$ and $\hat{a}^\dagger$.
Therefore,
\begin{align}
    \hat{x}(t) = \sqrt{\frac{1}{2\omega}}\left( \hat{a}(t) + \hat{a}^\dagger(t) \right) = \sqrt{\frac{1}{2\omega}}\left( \hat{a}e^{-iwt} + \hat{a}^\dagger e^{iwt} \right) \\
    \hat{p}(t) = -i\sqrt{\frac{\omega}{2}}\left( \hat{a}(t) - \hat{a}^\dagger(t) \right) = -i\sqrt{\frac{\omega}{2}}\left( \hat{a}e^{-iwt} - \hat{a}^\dagger e^{iwt} \right)
\end{align}
And also using the result of Prob.1(d)
\begin{align}
    H & = \omega \left( \hat{a}^\dagger(t)\hat{a}(t) + \frac{1}{2} \right) = \omega \left( \hat{a}^\dagger e^{iwt} \hat{a} e^{-iwt} + \frac{1}{2} \right) \\
    & = \omega \left( \hat{a}^\dagger\hat{a} + \frac{1}{2} \right)
\end{align}

\newpage

%%%%%%%%%%%%%%%%%%%%%%%%%%%%%%%%%%%%%%%%
%%%%%%%%%%%%%Problem 2.(a)%%%%%%%%%%%%%%
%%%%%%%%%%%%%%%%%%%%%%%%%%%%%%%%%%%%%%%%
\textbf{2. Review : Lorentz Transformation}
\begin{enumerate}
    \item [(a)] \textcolor{blue}{
    Prove that the four-dimensional $\delta$-function
    \begin{equation}\label{eqn:2.1}
        \delta^{(4)}(p) = \delta(p^0)\delta(p^1)\delta(p^2)\delta(p^3)
    \end{equation}
    is Lorentz invariant, i.e
    \begin{equation}\label{eqn:2.2}
        \delta^{(4)}(p) = \delta^{(4)}(\tilde{p})
    \end{equation}
    where $\tilde{p}^\mu$ is a Lorentz transformation of $p$.
    }
\end{enumerate}
%%%%%%%%%%%%%%%%%%%%%%%%%%%%%%%%%%%%%%%%%%%%%%%%%%%%%%%%%%%%%%%%%%%%%%%%%%%%%%%%
Let $\Lambda^\mu_\nu$ be a Lorentz transformation, then
\begin{equation}
      \tilde{p}^\mu = \Lambda^\mu_\nu p^\nu, \quad \tilde{x}^\mu = \Lambda^\mu_\nu x^\nu
\end{equation}
And then we have to keep in mind that $\Lambda x \cdot \Lambda p = x \cdot p$ is Lorentz invariant.
Then, we will make a Fourier transformation to the $\delta$-function
\begin{equation}
    \delta^{(4)}(p) = \frac{1}{(2\pi)^4} \int d^4 x e^{i x \cdot p} =  \frac{1}{(2\pi)^4} \int d^4 x e^{i \Lambda x \cdot \Lambda p}
\end{equation}
Since the property of the Lorentz transformation that:
\begin{align}
    \Lambda^T \eta \Lambda = \eta & \Rightarrow \det \Lambda^T \det \Lambda = (\det \Lambda)^2 = 1 \\
    & \Rightarrow J = |\det \Lambda| = 1
\end{align}
So,
\begin{equation}
    d^4\tilde{x} = d^4 x
\end{equation}
That makes the $\delta$-function,
\begin{equation}
     \frac{1}{(2\pi)^4} \int d^4 x e^{i \Lambda x \cdot \Lambda p} =  \frac{1}{(2\pi)^4} \int d^4 \tilde{x} e^{i \tilde{x} \cdot \tilde{p}}
\end{equation}
which is the $\delta$-function of $\tilde{p}$.

\newpage
%%%%%%%%%%%%%%%%%%%%%%%%%%%%%%%%%%%%%%%%
%%%%%%%%%%%%%Problem 2.(b)%%%%%%%%%%%%%%
%%%%%%%%%%%%%%%%%%%%%%%%%%%%%%%%%%%%%%%%
\begin{enumerate}
    \item [(b)] \textcolor{blue}{
    Show that
    \begin{equation}\label{eqn:2.3}
        \omega_1 \delta^{(3)}(\vec{k}_1 - \vec{k}_2)
    \end{equation}
    is Lorentz invariant, i.e
    \begin{equation}\label{eqn:2.4}
        \omega_1 \delta^{(3)}(\vec{k}_1 - \vec{k}_2) = \omega_1' \delta^{(3)}(\vec{k}'_1 - \vec{k}'_2)
    \end{equation}
    where $\vec{k}_1$ and $\vec{k}_2$ are respectively the spatial part of four-vectors $k^\mu_1 = (w_1, \vec{k}_1)$ and $k^\mu_2 = (w_2, \vec{k}_2)$ which satisfy the on-shell condition
    \begin{equation}\label{eqn:2.5}
        k^2_1 = k^2_2 = -m^2
    \end{equation}
    $k'^\mu_1  = (w'_1, \vec{k}'_1)$ and $k'^\mu_2 = (w'_2, \vec{k}'_2)$ are related to $k^\mu_1, k^\mu_2$ by a same Lorentz transformation.
    }
\end{enumerate}
%%%%%%%%%%%%%%%%%%%%%%%%%%%%%%%%%%%%%%%%%%%%%%%%%%%%%%%%%%%%%%%%%%%%%%%%%%%%%%%%
Since for the given condition that $k^\mu$s are in on-shell, it gives us the mass-shell condition
\begin{equation}
    \delta(k^2 + m^2)
\end{equation}
We will use the property of $\delta$-function to show the Lorentz invariance.
\begin{equation}
    \delta(f(x)) = \sum_{x_i = 0\  s.t \ f(x_i) = 0} \frac{1}{|f'(x_i)|}\delta(x-x_i)
\end{equation}
We modify the mass-shell condition;
\begin{align}
    \delta(k^2 + m^2) & = \delta(-k^2_0 + \vec{k}^2 + m^2) \\
    & = \delta(-k^2_0 + w^2_{\vec{k}}) \\
    & = \delta((-k_0 + |w_{\vec{k}}|)(k_0 + |w_{\vec{k}}|)) \\
    & = \frac{1}{2|\omega_{\vec{k}}|} \left[ \delta(k_0 - |\omega_{\vec{k}}|) + \delta(k^0 + |\omega_{\vec{k}}|) \right]
\end{align}
where $\omega_{\vec{k}}$ comes from the energy relationship, $\omega^2 = \vec{k}^2 + m^2$.
We assume that $\omega_{\vec{k}} > 0$\\
We can pick out the $k^0_1 = \omega_{\vec{k}_1}$ enforcing $\delta(k^2_1 + m^2)$ by multiplying both sides by $\theta(\omega_{\vec{k}_1})$.
\begin{align}
    \theta(\omega_{\vec{k}_1})\delta(k^2_1 + m^2) & = \frac{1}{2|\omega_{\vec{k}_1}|} \theta(\omega_{\vec{k}_1})\left[ \delta(k^0_1 - |\omega_{\vec{k}_1}|) + \delta(k^0_1 + |\omega_{\vec{k}_1}|) \right] \\
    & = \frac{1}{2\omega_{\vec{k}_1}} \theta(\omega_{\vec{k}_1})\delta(k^0_1 - \omega_{\vec{k}_1}) \\
    & = \frac{1}{2\omega_{\vec{k}_1}} \delta(k^0_1 - \omega_{\vec{k}_1})
\end{align}
Then, we multiply both sides by $2\omega_{\vec{k}_1} \delta^{(3)}(\vec{k}_1 - \vec{k}_2)$
\begin{align}
    2\omega_{\vec{k}_1} \delta^{(3)}(\vec{k}_1 - \vec{k}_2)\theta(\omega_{\vec{k}_1})\delta(k^2_1 + m^2) & = 2\omega_{\vec{k}_1} \delta^{(3)}(\vec{k}_1 - \vec{k}_2) \cdot \frac{1}{2\omega_{\vec{k}}} \delta(k^0_1 - \omega_{\vec{k}_1}) \\
    & = \delta^{(3)}(\vec{k}_1 - \vec{k}_2)\delta(k^0_1 - \omega_{\vec{k}_1}) \\
    & = \delta^{(3)}(\vec{k}_1 - \vec{k}_2)\delta(k^0_1 - \omega_{\vec{k}_2}) \\
    & = \delta^{(3)}(\vec{k}_1 - \vec{k}_2)\delta(k^0_1 - k^0_2)
\end{align}

In eqn.(77), we use that the $\delta^{(3)}(\vec{k}_1 - \vec{k}_2)$ allows us to replace $\omega_{\vec{k}_1}$ to $\omega_{\vec{k}_2}$
For this step, it is crucial that $sign(\omega_{\vec{k}_1}) = sign(\omega_{\vec{k}_2})$, which is true since both are positive.

Finally, from eqn.(78), the right-handed side is Lorentz invariant, since we show that in Prob.2(a).
On the left-handed side, we already know that $\delta(k^2_1 + m^2)$ is a Lorentz scalar since $k^2_1$ is Lorentz invariant, and $\theta(\omega_{\vec{k}_1})$ is Lorentz invariant since the energy of a particle does not change under Lorentz transformation. 
Therefore, we can say that $\omega_{\vec{k}_1}\delta^{(3)}(\vec{k}_1 - \vec{k}_2)$ is Lorentz invariant to satisfy both sides.

%%%%%%%%%%%%%%%%%%%%%%%%%%%%%%%%%%%%%%%%
%%%%%%%%%%%%%Problem 2.(c)%%%%%%%%%%%%%%
%%%%%%%%%%%%%%%%%%%%%%%%%%%%%%%%%%%%%%%%
\begin{enumerate}
    \item [(c)] \textcolor{blue}{
    For any function $f(k) = f(k^0, k^1, k^2, k^3)$, prove that
    \begin{equation}\label{eqn:2.6}
        \int \frac{d^3 \vec{k}}{(2\pi)^3} \frac{1}{2\omega_{\vec{k}}} f(k), \quad \omega_{\vec{k}} = \sqrt{\vec{k}^2 + m^2}
    \end{equation}
    is Lorentz invariant in the sense that
    \begin{equation}\label{eqn:2.7}
        \int \frac{d^3 \vec{k}}{(2\pi)^3} \frac{1}{2\omega_{\vec{k}}} f(k) = \int \frac{d^3 \vec{k}}{(2\pi)^3} \frac{1}{2\omega_{\vec{k}}} f(\tilde{k})
    \end{equation}
    where $\tilde{k}^\mu = \Lambda^\mu_\nu k^\nu$ is a Lorentz transformation of $k^\mu$
    }
\end{enumerate}
%%%%%%%%%%%%%%%%%%%%%%%%%%%%%%%%%%%%%%%%%%%%%%%%%%%%%%%%%%%%%%%%%%%%%%%%%%%%%%%%
Since the momentum is on the mass-shell, $f(k) = f(\omega_{\vec{k}},\vec{k})$
So, the integral over the spatial part can be written as;
\begin{align}
    \int \frac{d^3 \vec{k}}{(2\pi)^3} \frac{1}{2\omega_{\vec{k}}} f(k) & = \int \frac{d^3 \vec{k}}{(2\pi)^3} \frac{1}{2\omega_{\vec{k}}} f(\omega_{\vec{k}},\vec{k}) \\
    & = \int \frac{d^3 \vec{k}}{(2\pi)^3} \frac{1}{2\omega_{\vec{k}}} \delta(k^0 - \omega_{\vec{k}}) f(k^0,\vec{k}) \\
    & = \frac{1}{(2\pi)^3}\int d^4k \  \theta(\omega_{\vec{k}})\delta(k^2 + m^2) f(k^0,\vec{k})
\end{align}
Then, Lorentz transformation follows the property below:
$$
d^4 k = d^4\tilde{k}, \quad \theta(\omega_{\vec{k}}) = \theta(\omega_{\vec{k}'}), \quad \delta(k^2+m^2) = \delta(k'^2+m^2)
$$
The integral is noted as:
\begin{align}
    \int \frac{d^3 \vec{k}}{(2\pi)^3} \frac{1}{2\omega_{\vec{k}}} f(k) & = \frac{1}{(2\pi)^3}\int d^4k \  \theta(\omega_{\vec{k}})\delta(k^2 + m^2) f(k^0,\vec{k}) \\
    & = \frac{1}{(2\pi)^3}\int d^4k' \  \theta(\omega_{\vec{k}'})\delta(k'^2 + m^2) f((\Lambda k')^\mu) \\
    & = \int \frac{d^3\vec{k}}{(2\pi)^3} \frac{1}{2\omega_{\vec{k}}} f(\Lambda k) \\
    & = \int \frac{d^3\vec{k}}{(2\pi)^3} \frac{1}{2\omega_{\vec{k}}} f(\tilde{k})
\end{align}
So, $\int \frac{d^3 \vec{k}}{(2\pi)^3} \frac{1}{2\omega_{\vec{k}}} f(k)$ is Lorentz invariant.

\newpage

%%%%%%%%%%%%%%%%%%%%%%%%%%%%%%%%%%%%%%%%
%%%%%%%%%%%%%%%Problem 3%%%%%%%%%%%%%%%%
%%%%%%%%%%%%%%%%%%%%%%%%%%%%%%%%%%%%%%%%
\textbf{3. A Complex Scalar field}\\
\textcolor{blue}{
Consider the field theory of a complex value scalar field $\phi(x)$ with action
\begin{equation}\label{eqn:3.1}
    S = \int d^4 x \ \left[ -\partial_\mu \phi^* \partial^\mu \phi - V(|\phi|^2) \right], \quad |\phi|^2 = \phi \phi^*
\end{equation}
One could either consider the real and imaginary parts of $\phi$, or $\phi$ and $\phi^*$ as independent dynamical variables. The latter is more convenient in most situations.
}
%%%%%%%%%%%%%%%%%%%%%%%%%%%%%%%%%%%%%%%%
%%%%%%%%%%%%%Problem 3.(a)%%%%%%%%%%%%%%
%%%%%%%%%%%%%%%%%%%%%%%%%%%%%%%%%%%%%%%%
\begin{enumerate}
    \item [(a)] \textcolor{blue}{
    Check action (\ref{eqn:3.1}) is Lorentz invariant ($\phi(x) \to \phi'(x') = \phi(x)$) and find the equations of motion.
    }
\end{enumerate}
%%%%%%%%%%%%%%%%%%%%%%%%%%%%%%%%%%%%%%%%%%%%%%%%%%%%%%%%%%%%%%%%%%%%%%%%%%%%%%%%
Lorentz transformation acts as $\phi \to \phi^*$, such that $\phi'(x) = \phi(\Lambda^{-1} x)$
Then the action transforms like;
\begin{align}
    S  \to S' & = \int d^4 x \ \left[ -\partial_\mu \phi'^*(x) \partial^\mu \phi'(x) - V(|\phi'(x)|^2) \right] \\
    & = \int d^4 x \ \left[ -\partial_\mu \phi^*(\Lambda^{-1}x) \partial^\mu \phi(\Lambda^{-1}x) - V(|\phi(\Lambda^{-1}x)|^2) \right]
\end{align}
From the Prob.2(a), we know that $d^4x = d^4x'$.
Use the chain rule for modifying the derivative that $\partial_\mu = (\Lambda^{-1})^\nu_\mu\partial'_\nu$.
\begin{align}
    S' & = \int d^4 x' \ \left[-(\Lambda^{-1})^\nu_\mu\partial'_\nu \phi^*(x') (\Lambda^{-1})^\mu_\rho\partial'^\rho \phi(x') - V(|\phi(x')|^2) \right] \\
    & = \int d^4 x' \ \left[-( \Lambda^{-1})^\nu_\mu(\Lambda^{-1})^\mu_\rho \partial'_\nu \phi^*(x') \partial'^\rho \phi(x') - V(|\phi(x')|^2) \right] \\
    & = \int d^4 x' \ \left[-( \delta^\nu_\rho \partial'_\nu \phi^*(x') \partial'^\rho \phi(x') - V(|\phi(x')|^2) \right] \\
    & = \int d^4 x' \ \left[-( \partial'_\nu \phi^*(x') \partial'^\nu \phi(x') - V(|\phi(x')|^2) \right] \\
    & = S
\end{align}
where we can use the property of the Lorentz transformation:
$$ (\Lambda^{-1})^\nu_\mu(\Lambda^{-1})^\mu_\rho = ((\Lambda^{-1})^T)^\nu_\mu(\Lambda^{-1})^\mu_\rho = \Lambda^\nu_\mu(\Lambda^{-1})^\mu_\rho = \delta^\nu_\rho$$
Therefore, the action is Lorentz invariant. \\
For the equation of motions, we have to calculate Euler-Lagrangian equation for $\phi$ and $\phi^*$ independently
\begin{equation}
\begin{cases}
     \partial_\mu \frac{\partial \mathcal{L}}{\partial (\partial_\mu \phi)} = \frac{\partial \mathcal{L}}{\partial \phi} & \mbox{(i)}\\
     \\
    \partial_\mu \frac{\partial \mathcal{L}}{\partial (\partial_\mu \phi^*)} = \frac{\partial \mathcal{L}}{\partial \phi^*} & \mbox{(ii)}
\end{cases}
\end{equation}
where $\mathcal{L} =  -\partial_\mu \phi'^*(x) \partial^\mu \phi'(x) - V(|\phi'(x)|^2 $, for $\mbox{(i)}$;
\begin{align}
     \partial_\mu \frac{\partial \mathcal{L}}{\partial (\partial_\mu \phi)} & = \partial_\mu \frac{\partial}{\partial (\partial_\mu \phi)}\left[ -\partial_\nu \phi^* \partial^\nu \phi - V(|\phi|^2) \right] \\
    & = \partial_\mu (-\partial_\nu \phi^* \delta^\nu_\mu) \\
    & = -\partial^2\phi^* \\
    \frac{\partial \mathcal{L}}{\partial \phi} & = \frac{\partial}{\partial \phi} (-V(|\phi|^2) ) \\
    & = -V'(|\phi|^2) \phi^*
\end{align}
Then, by Euler-Lagrangian equation
\begin{equation}
    \partial^2\phi^* -V'(|\phi|^2) \phi^* = 0
\end{equation}
Same sequence for $\mbox{(ii)}$, we can get
\begin{equation}
    \partial^2\phi -V'(|\phi|^2) \phi = 0
\end{equation}

%%%%%%%%%%%%%%%%%%%%%%%%%%%%%%%%%%%%%%%%
%%%%%%%%%%%%%Problem 3.(b)%%%%%%%%%%%%%%
%%%%%%%%%%%%%%%%%%%%%%%%%%%%%%%%%%%%%%%%
\begin{enumerate}
    \item [(b)] \textcolor{blue}{
    Find the cancnonical conjugate momenta for $\phi$ and $\phi^*$, and the Hamiltonian $H$ for eqn.(\ref{eqn:3.1})
    }
\end{enumerate}
%%%%%%%%%%%%%%%%%%%%%%%%%%%%%%%%%%%%%%%%%%%%%%%%%%%%%%%%%%%%%%%%%%%%%%%%%%%%%%%%
We can expand the 4-derivative$\partial_\mu$ to time and space derivative, that makes the Lagrangian $\mathcal{L}$ into :
\begin{equation}
    \mathcal{L} = \partial_t \phi^* \partial^t \phi - \vec{\nabla}\phi^* \cdot \vec{\nabla}\phi - V(|\phi|^2)
\end{equation}
Therefore, the canonical conjugate momenta becomes:
\begin{equation}
    \pi := \frac{\partial \mathcal{L}}{\partial (\partial_t \phi)} = \partial_t \phi^*, \quad \pi^* := \frac{\partial \mathcal{L}}{\partial (\partial_t \phi^*)} = \partial_t \phi
\end{equation}
So the Hamiltonian is:
\begin{align}
    H & = \int d^3 x \left[ \pi \cdot \partial_t\phi +\pi^* \cdot \partial_t\phi^* - \mathcal{L} \right] \\
    & = \int d^3 x \left[ 2 \pi^*\pi - \pi^*\pi + \vec{\nabla}\phi^* \cdot \vec{\nabla}\phi + V(|\phi|^2\right] \\
    & = \int d^3 x \left[ \pi^*\pi + \vec{\nabla}\phi^* \cdot \vec{\nabla}\phi + V(|\phi|^2) \right]
\end{align}

\newpage

%%%%%%%%%%%%%%%%%%%%%%%%%%%%%%%%%%%%%%%%
%%%%%%%%%%%%%Problem 3.(c)%%%%%%%%%%%%%%
%%%%%%%%%%%%%%%%%%%%%%%%%%%%%%%%%%%%%%%%
\begin{enumerate}
    \item [(c)] \textcolor{blue}{
    The action is invariant under transformation
    \begin{equation}\label{eqn:3.2}
        \phi \to e^{i\alpha} \phi, \quad \phi^* \to e^{-i\alpha} \phi^*
    \end{equation}
    for arbitrary constant $\alpha$. When $\alpha$ is small, $\mbox{i.e.}$ for an infinitesimal transformation, eqn.(\ref{eqn:3.2}) become
    \begin{equation}\label{eqn:3.3}
        \delta\phi = i\alpha\phi, \quad \delta\phi^* = -i\alpha \phi^*
    \end{equation}
    Use Noether's theorem to find the corresponding conserved current $j^\mu$ and conserved change $Q$.
    }
\end{enumerate}
%%%%%%%%%%%%%%%%%%%%%%%%%%%%%%%%%%%%%%%%%%%%%%%%%%%%%%%%%%%%%%%%%%%%%%%%%%%%%%%%
Noether's theorem states that very continuous symmetry of the action of a physical system with conservative forces has a corresponding conservation law.
So that conserve current is:
\begin{equation}
    j^\mu = \frac{\partial \mathcal{L}}{\partial (\partial_\mu \Phi_{a})}\delta\Phi_{a} - \mathcal{F}^\mu \quad \mbox{where} \quad \delta \mathcal{L} = \partial_\mu \mathcal{F}^\mu
\end{equation}
Then, use the infinitesimal transformation the conserved current is
\begin{align}
    j^\mu & = -\partial^\mu \phi^*(i\alpha\phi) - \partial^\mu \phi (-i\alpha\phi^*) \\
    & = i\alpha(\phi^* \partial^\mu \phi - \phi \partial^\mu \phi^*)
\end{align}
we can remove that proportional constant to simplify the current
\begin{equation}
    j^\mu = \phi^* \partial^\mu \phi - \phi \partial^\mu \phi^*
\end{equation}
Then the corresponding charge is
\begin{equation}
    Q = \int d^3x \ j^0 = \int d^3 x (\phi^* \partial_t \phi - \phi \partial_t \phi^*)
\end{equation}

%%%%%%%%%%%%%%%%%%%%%%%%%%%%%%%%%%%%%%%%
%%%%%%%%%%%%%Problem 3.(d)%%%%%%%%%%%%%%
%%%%%%%%%%%%%%%%%%%%%%%%%%%%%%%%%%%%%%%%
\begin{enumerate}
    \item [(d)] \textcolor{blue}{
    Use the equations of motion of part (a) to verify directly that $j^\mu$ is indeed conserved.
    }
\end{enumerate}
%%%%%%%%%%%%%%%%%%%%%%%%%%%%%%%%%%%%%%%%%%%%%%%%%%%%%%%%%%%%%%%%%%%%%%%%%%%%%%%%
Using the equation of motion, the conservation of the current can be checked with caculating the derivative:
\begin{align}
    \partial_\mu j^\mu & = \partial_\mu  (\phi^* \partial_t \phi - \phi \partial_t \phi^*) \\
    & = \phi^* \partial^2\phi - \phi \partial^2 \phi^* \\
    & = V'(|\phi|^2)\phi^*\phi -  V'(|\phi|^2)\phi^*\phi \\
    & = 0
\end{align}

\newpage

%%%%%%%%%%%%%%%%%%%%%%%%%%%%%%%%%%%%%%%%
%%%%%%%%%%%%%%%Problem 4%%%%%%%%%%%%%%%%
%%%%%%%%%%%%%%%%%%%%%%%%%%%%%%%%%%%%%%%%
\textbf{4. The energy-momentum tensor for the complex scalar field theory}\\
\textcolor{blue}{
In this problem, we work out the energy-momentum tensor of the complex scalar theory (\ref{eqn:3.1}).
}
%%%%%%%%%%%%%%%%%%%%%%%%%%%%%%%%%%%%%%%%
%%%%%%%%%%%%%Problem 4.(a)%%%%%%%%%%%%%%
%%%%%%%%%%%%%%%%%%%%%%%%%%%%%%%%%%%%%%%%
\begin{enumerate}
    \item [(a)] \textcolor{blue}{
    Under a spacetime translation
    \begin{equation}\label{eqn:4.1}
        x^\mu \to x'^\mu = x^\mu + a^\mu
    \end{equation}
    a scalar field transform as
    \begin{equation}\label{eqn:4.2}
        \phi'(x') = \phi(x)
    \end{equation}
    Show that the action (\ref{eqn:3.1}) is invariant under transformation $\phi(x) \to \phi'(x)$.
    }
\end{enumerate}
%%%%%%%%%%%%%%%%%%%%%%%%%%%%%%%%%%%%%%%%%%%%%%%%%%%%%%%%%%%%%%%%%%%%%%%%%%%%%%%%
Under the transformation, the scalar field satisfies $\phi'(x) = \phi(x-a)$.
Then the action transforms:
\begin{align}
    S \to S' & = \int d^4 x (-\partial_\mu \phi'^*(x) \partial^\mu \phi'(x) - V(|\phi'(x)|^2) \\
    & = \int d^4x  (-\partial_\mu \phi^*(x-a) \partial^\mu \phi(x-a) - V(|\phi(x-a)|^2) \\
    & = \int d^4x  (-\partial_\mu \phi^*(x) \partial^\mu \phi(x) - V(|\phi(x)|^2) = S
\end{align}

%%%%%%%%%%%%%%%%%%%%%%%%%%%%%%%%%%%%%%%%
%%%%%%%%%%%%%Problem 4.(b)%%%%%%%%%%%%%%
%%%%%%%%%%%%%%%%%%%%%%%%%%%%%%%%%%%%%%%%
\begin{enumerate}
    \item [(b)] \textcolor{blue}{
    Write down the transformation of the scalar fields $\phi$ and $\phi^*$ for an infinitesimal translation, and use Noether's theorem to find the corresponding conserved currents $T^{\mu\nu}$.
    }
\end{enumerate}
%%%%%%%%%%%%%%%%%%%%%%%%%%%%%%%%%%%%%%%%%%%%%%%%%%%%%%%%%%%%%%%%%%%%%%%%%%%%%%%%
An infinitesimal transformation makes the field:
\begin{align}
    \delta\phi & = \phi'(x) - \phi(x) = \phi(x-a) - \phi(x) = -a^\mu \partial_\mu \phi(x)\\
    \delta\phi^* & = \phi'^*(x) - \phi^*(x) = \phi^*(x-a) - \phi^*(x) = -a^\mu \partial_\mu \phi^*(x)
\end{align}
Then, the Lagrangian density under translation gives the infinitesimal variance:
\begin{align}
    \delta\mathcal{L} & = \mathcal{L}' - \mathcal{L} \\
    & = - \partial_\nu(\phi^*(x) - a^\mu\partial_\mu\phi^*(x)) \partial^\nu (\phi(x) - a^\mu\partial_\mu\phi(x)) \\
    & \quad \quad \quad \quad \quad \quad \quad \quad \quad \quad \quad \quad \quad \quad 
    - V\left( (\phi^*(x) - a^\mu\partial_\mu\phi^*(x)(\phi(x) - a^\mu\partial_\mu\phi(x)))  \right) - \mathcal{L} \\
    & = a^\mu(\partial_\nu\partial_\mu \phi^*(x) \partial^\nu \phi(x) + \partial_\nu \phi^* \partial^\nu \partial_\mu \phi(x)) + a^\mu V'(|\phi|^2)(\partial_\mu\phi^* \phi+\phi^* \partial_\mu\phi) + \mathcal{O}(a^\mu a^\nu) \\
    & = -a^\mu \partial_\mu \mathcal{L} \\
    & = a_\mu \partial_\nu (-\eta^{\mu\nu} \mathcal{L}) := a_\mu \partial_\nu \mathcal{F}^{\mu\nu}
\end{align}
Translations are parameterize with 4-vector $a^\mu$, and we have a Noether current itself a 4-vector.
Therefore, the conserved currents are encoded in the 2-rank tensor, $\mathcal{T}^{\mu\nu}$.
\begin{align}
    T^{\mu\nu} := (j^\mu)^\nu & = \frac{\partial \mathcal{L}}{\partial (\partial_\mu \phi)}(\delta\phi)^\mu + \frac{\partial \mathcal{L}}{\partial (\partial_\mu \phi^*)}(\delta\phi^*)^\mu - \mathcal{F}^{\mu\nu} \\
    & = - \partial^\nu \phi^* (-\partial^\mu \phi) - \partial^\nu\phi(\partial^\mu\phi^*) + \eta^{\mu\nu}\mathcal{L} \\
    & =  \partial^\nu \phi^* \partial^\mu \phi + \partial^\nu\phi \partial^\mu\phi^* - \eta^{\mu\nu}\left[\partial_\rho \phi^* \partial^\rho \phi + V(|\phi|^2) \right]
\end{align}
In the following, we let the first index $\mu$ of $T^{\mu\nu}$ pick out the direction of translation $a^\mu$

\newpage

%%%%%%%%%%%%%%%%%%%%%%%%%%%%%%%%%%%%%%%%
%%%%%%%%%%%%%Problem 4.(c)%%%%%%%%%%%%%%
%%%%%%%%%%%%%%%%%%%%%%%%%%%%%%%%%%%%%%%%
\begin{enumerate}
    \item [(c)] \textcolor{blue}{
    The conserved charge for a time translation
    \begin{equation}\label{eqn:4.3}
        H = \int d^3x T^{00}
    \end{equation}
    should be identified with the total energy of the system, while that for a spatial translation
    \begin{equation}\label{eqn:4.4}
        P^i = \int d^3x T^{0i}
    \end{equation}
    should be identified with the total momentum. Thus, $T^{\mu\nu}$ is referred to as the energy-momentum tensor. Write down the explicit expression for $H$ and $P^i$. Compare $H$ obtained here with the Hamiltonian of prob.3(b).
    }
\end{enumerate}
%%%%%%%%%%%%%%%%%%%%%%%%%%%%%%%%%%%%%%%%%%%%%%%%%%%%%%%%%%%%%%%%%%%%%%%%%%%%%%%%
The Hamiltonian is:
\begin{align}
    H & = \int d^3x T^{00} \\
    & = \int d^3x \left[ 2\partial^t \phi^* \partial^t \phi - \eta^{00}(\partial_\mu \phi^* \partial^\mu \phi + V(|\phi|^2)) \right] \\
    & = \int d^3x \left[ 2\partial^t \phi^* \partial^t \phi + (-\partial_t \phi^* \partial_t \phi + \vec{\nabla}\phi^* \cdot \vec{\nabla}\phi + V(|\phi|^2)) \right] \\
    & =\int d^3x \left[ \partial^t \phi^* \partial^t \phi + \vec{\nabla}\phi^* \cdot \vec{\nabla}\phi + V(|\phi|^2)) \right]
\end{align}
And the total momentum is:
\begin{align}
    P^{i} & = \int d^3x T^{0i} \\
    & = \int d^3x \left[ \partial^t \phi^* \partial^i \phi + \partial^i\phi^* \partial^t\phi - \eta^{0i}(\partial_\rho \phi^* \partial^\rho \phi + V(|\phi|^2) ) \right] \\
    & = \int d^3x \left[ \partial^t \phi^* \partial^i \phi + \partial^i\phi^* \partial^t\phi \right] \\
    & = - \int d^3x \left[ \partial_t \phi^* \partial_i \phi + \partial_i\phi^* \partial_t\phi \right]
\end{align}
The expression of Hamiltonian is same as prob.3(b).

%%%%%%%%%%%%%%%%%%%%%%%%%%%%%%%%%%%%%%%%
%%%%%%%%%%%%%Problem 4.(d)%%%%%%%%%%%%%%
%%%%%%%%%%%%%%%%%%%%%%%%%%%%%%%%%%%%%%%%
\begin{enumerate}
    \item [(d)] \textcolor{blue}{
    Use equations of motion of prob.3(a) to verify directly that $T^{\mu\nu}$ is indeed conserved.
    }
\end{enumerate}
%%%%%%%%%%%%%%%%%%%%%%%%%%%%%%%%%%%%%%%%%%%%%%%%%%%%%%%%%%%%%%%%%%%%%%%%%%%%%%%%
The equations of motion are:
\begin{equation}
    \begin{cases}
        \partial^2\phi^* - V'(|\phi|^2) \phi^* = 0 \\
        \partial^2\phi - V'(|\phi|^2) \phi = 0
    \end{cases}
\end{equation}
Recall that $\mu$ indicates the direction of translation $a^\mu$.
Therefore, conservation from the Noether's theorem means that $\partial_\nu T^{\mu\nu}$.
Caring the results from prob.4(c), $T^{\mu\nu}$ is symmetric so that we can contract the derivative with respect to either index.
\begin{align}
    \partial_\mu T^{\mu\nu} & = \partial_\mu(\partial^\nu \phi^* \partial^\mu \phi + \partial^\mu\phi^*\partial^\nu\phi - \eta^{\mu\nu}\left[\partial_\rho \phi^* \partial^\rho \phi + V(|\phi|^2) \right]) \\
    & = \partial_\mu(\partial^\nu \phi^* \partial^\mu \phi + \partial^\mu\phi^*\partial^\nu\phi ) -\partial^\nu(\partial_\rho \phi^* \partial^\rho \phi) - \partial^\nu V(|\phi|^2) \\
    & = \partial^2\phi^* \partial^\nu\phi + \partial^\nu\phi^* \partial^2\phi - \partial^\nu V(|\phi|^2) \\
    & = V'(|\phi|^2) \phi^* \partial^\nu\phi + \partial^\nu\phi^* V'(|\phi|^2) \phi - V'(|\phi|^2) \partial^\nu\phi^* \phi - V'(|\phi|^2) \phi^* \partial^\nu \phi \\
    & = 0
\end{align}

\end{document}

